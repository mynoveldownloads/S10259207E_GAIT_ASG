\documentclass[12pt]{article}
\usepackage[margin=1in]{geometry}
\usepackage{amsmath,amssymb}
\usepackage{enumitem}
\usepackage{tikz}
\usepackage{tcolorbox}
\tcbuselibrary{skins,breakable}
\usepackage{hyperref}
\hypersetup{
    colorlinks=true,
    linkcolor=blue,
    urlcolor=blue
}
\title{Lecture Highlights}
\author{Compiled by Academic Assistant}
\date{\today}

%--- Box styles -------------------------------------------------
\newtcolorbox{keytakeaway}{
    colback=blue!5!white,
    colframe=blue!75!black,
    title=Key Takeaway,
    fonttitle=\bfseries,
    breakable
}
\newtcolorbox{remember}{
    colback=green!5!white,
    colframe=green!75!black,
    title=Remember,
    fonttitle=\bfseries,
    breakable
}
\newtcolorbox{crucialinsight}{
    colback=yellow!5!white,
    colframe=yellow!75!black,
    title=Crucial Insight,
    fonttitle=\bfseries,
    breakable
}
\newtcolorbox{codebox}{
    colback=gray!10!white,
    colframe=black,
    title=Code,
    fonttitle=\bfseries,
    breakable
}
%--- End box styles ---------------------------------------------

\begin{document}
\maketitle

\section*{Overview}
The excerpt presented is a famous opening line from a classic novel.  
It juxtaposes opposite ideas to illustrate the complexity of an era.

\section{Literary Context}
\subsection*{Definition}
\begin{keytakeaway}
A literary opening that places contrasting statements side by side.
\end{keytakeaway}
\subsection*{Technical Definition}
\begin{remember}
The sentence employs \textbf{antithesis} – a rhetorical device that puts two opposite ideas together for effect.
\end{remember}
\subsection*{Purpose}
\begin{crucialinsight}
The purpose is to immediately convey the dual nature of the period being described, preparing the reader for a story that explores both extremes.
\end{crucialinsight}
\subsection*{Why This Method Is Chosen}
\begin{crucialinsight}
Antithesis is chosen because it creates a strong emotional hook; readers can feel the tension between “best” and “worst” without needing detailed exposition.
\end{crucialinsight}
\subsection*{Advantages}
\begin{itemize}[label=$\bullet$]
    \item Captures attention instantly.
    \item Sets a thematic tone for the entire work.
    \item Requires few words to convey a broad spectrum of ideas.
\end{itemize}
\subsection*{Disadvantages}
\begin{itemize}[label=$\bullet$]
    \item May oversimplify complex historical realities.
    \item Risks sounding melodramatic if overused.
\end{itemize}
\subsection*{Role During Reading}
\begin{enumerate}[label=\arabic*.]
    \item \textbf{First impression}: establishes expectations.
    \item \textbf{Guiding lens}: readers interpret later events through the lens of contrast.
\end{enumerate}
\subsection*{Interaction With Other Narrative Elements}
The antithetical opening interacts with character development, plot twists, and setting descriptions, reinforcing the theme of duality throughout the narrative.

\section{Visual Representation}
\subsection*{Diagram of Contrasting Themes}
\begin{tikzpicture}[node distance=1.5cm, auto, font=\small]
    \node[ellipse, draw, fill=blue!10] (start) {Start};
    \node[rectangle, draw, right=of start, fill=yellow!10] (best) {Best of Times};
    \node[rectangle, draw, below=of best, fill=yellow!10] (worst) {Worst of Times};
    \node[diamond, draw, right=of best, aspect=2, fill=green!10] (choice) {Contrast?};
    \node[ellipse, draw, right=of choice, fill=blue!10] (end) {End};

    \draw[->] (start) -- (best);
    \draw[->] (start) -- (worst);
    \draw[->, thick, red] (best) -- (choice);
    \draw[->, thick, red] (worst) -- (choice);
    \draw[->] (choice) -- (end);
\end{tikzpicture}
\begin{crucialinsight}
The diagram shows how two opposite ideas feed into a central decision point that shapes the narrative direction.
\end{crucialinsight}

\section{Common Implementation Errors}
\begin{itemize}[label=$\bullet$]
    \item Using a dash instead of a hyphen can break LaTeX compilation; always use the standard hyphen character.
    \item Forgetting to close a \texttt{tcolorbox} environment leads to compilation errors.
    \item Overloading the title page with too much text can cause formatting overflow.
\end{itemize}

\section{Extensions and Advanced Topics}
\subsection*{Rhetorical Devices Beyond Antithesis}
Exploring \textbf{paradox}, \textbf{irony}, and \textbf{oxymoron} can deepen the thematic richness of an opening line.

\subsection*{Adaptation to Modern Media}
Translating such contrasting openings to visual storytelling (film, graphic novels) requires careful framing and pacing to preserve the original impact.

\end{document}